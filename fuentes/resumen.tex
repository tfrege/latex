\chapter*{RESUMEN}

El control de calidad es la clave para obtener productos altamente competitivos en el mercado mundial, y en consecuencia, es tambi�n la clave del �xito de una empresa.\\

En el desarrollo de software, este control es una tarea compleja que engloba un conjunto de actividades que deben ser ejecutadas a lo largo de todo el proceso de desarrollo por personas dedicadas espec�ficamente a esta labor.\\

Para el presente proyecto se propuso la elaboraci�n de un documento que sea el resultado de la investigaci�n de las distintas t�cnicas, modelos y est�ndares de control de calidad que existen en la actualidad en el �rea de la ingenier�a del software y el desarrollo de un sistema que, basado en el resultado de la investigaci�n, permita la sistematizaci�n de las herramientas m�s �tiles que ayuden a las empresas de software en sus tareas de control de calidad.\\

Como resultado de este planteamiento, el presente documento ofrece una s�ntesis de la investigaci�n realizada en el campo de la Ingenier�a de Calidad del Software, presentando los distintos puntos de vista de los diferentes modelos y est�ndares que existen actualmente, como tambi�n las recomendaciones generales para la implementaci�n de pol�ticas de control de calidad en empresas dedicadas al desarrollo de software y de las causas y consecuencias que tienen los distintos problemas que surgen al no existir estas pol�ticas en las organizaciones.\\

Asimismo, se presenta el sistema desarrollado en base al {\it Aseguramiento de la Calidad del Software (SQA)} que facilita el proceso de control de calidad durante el desarrollo del software mediante las t�cnicas recomendadas por el SQA, la estructuraci�n y la sistematizaci�n del proceso completo de control (desde las fases m�s tempranas del desarrollo hasta las �ltimas).\\

{\bfseries Palabras clave:} Calidad del Software, SQA, Pruebas de Software, Control, M�trica, Marcos de Trabajo, Est�ndar, Herramienta, Caso de Prueba, Proceso, Madurez, Defecto.

\newpage
