\chapter*{GLOSARIO}
\addcontentsline{toc}{chapter}{GLOSARIO}

{\bfseries Algoritmo}.- Secuencia finita de instrucciones dada para resolver un problema espec�fico. Cada una de estas instrucciones tiene un significado claro y es ejecutada con una cantidad de esfuerzo finita.\\

{\bfseries ANSI}.- El ``American National Standards Institute'', al igual que el SEI, la ISO y el IEEE, est� vinculado con la calidad mediante sus publicaciones de est�ndares.\\
		
{\bfseries API}.-	Application Programming Interface. Conjunto de rutinas o funciones que un programa puede utilizar para que el Sistema Operativo realice alguna tarea.\\
		
{\bfseries Caja Blanca, Testeo de}.- Conjunto de pruebas donde el tester es responsable de verificar la ejecuci�n de cada prueba hasta el m�s m�nimo nivel de detalle, incluyendo una revisi�n del c�digo involucrado.\\
		
{\bfseries Caja Gris, Testeo de}.- H�brido entre el testeo de caja blanca y el testeo de caja negra. El tester verifica el c�digo sin mucho nivel de detalle.\\
		
{\bfseries Caja Negra, Testeo de}.- Conjunto de pruebas donde lo �nico que interesa es verificar la funcionalidad y no as� la manera en que el programa ejecuta las tareas. Al tester no le interesa si el c�digo est� bien escrito, simplemente si la tarea es bien realizada o no.\\
		
{\bfseries CGI}.- Common Gateway Interface. Es un protocolo gen�rico que permite extender las capacidades de HTTP. Los programas CGI a�aden funcionalidad al servidor Web.\\
		
{\bfseries Cliente}.- Ordenador que funciona localmente y se comunica con el servidor para solicitar informaci�n.\\
		
{\bfseries Defecto}.- Anomal�a de un producto.\\
		
{\bfseries EIS}.- Enterprise Information System. Se refiere a cualquier sistema que proporciona una infraestructura para el soporte de la informaci�n de la empresa.\\
		
{\bfseries Escalabilidad}.- Es la medida en que un sistema es capaz de soportar y servir a m�s de un usuario, entendi�ndose como usuario a cualquier agente que haga uso del sistema, ya sea una persona como otro sistema.\\
		
{\bfseries Framework}.- Es la extensi�n de un lenguaje mediante una o m�s jerarqu�as de clases que implementan una funcionalidad y que (opcionalmente) pueden ser extendidas.\\
		
{\bfseries HTML}.- HiperText Markup Language. Conjunto de marcas (m�s conocidas como etiquetas o "tags" en ingl�s) que permite b�sicamente el formateo de documentos de hipertexto para su divulgaci�n en el World Wide Web.\\
		
{\bfseries HTTP}.- HiperText Transfer Protocol. Principal protocolo de Internet para la transferencia de archivos.\\
		
{\bfseries IEEE}.- ``Institute of Electrical and Electronics Engineers''. Entre otras publicaciones, se encuentra relacionado a la calidad del software mediante sus est�ndares ``IEEE Standard for Software Test Documentation'' (IEEE/ANSI 829), ``IEEE Standard of Software Unit Testing'' (IEEE/ANSI Standard 1008) y ``IEEE Standard for Software Quality Assurance Plans'' (IEEE/ANSI Standard 730).\\
		
{\bfseries IIS}.- El Internet Information Server es el conjunto de herramientas ofrecidas por Microsoft para la administraci�n y control de sitios Web, FTP, SMTP y Servicio de noticias. 
		
{\bfseries Interface}.- Parte de una aplicaci�n capaz de interactuar con el usuario.\\
		
{\bfseries ISO}.- ``Internacional Organization for Standarization''. Esta organizaci�n est� involucrada con la calidad mediante sus est�ndares 9001, 9002 y 9003.\\
		
{\bfseries L�gica del Negocio}.- Porci�n de la arquitectura realizada por componentes que es utilizada para crear y garantizar las reglas establecidas.\\
		
{\bfseries Proceso}.- Un proceso define qui�n hace qu�, cu�ndo y c�mo para lograr un objetivo. Es una secuencia de tareas b�sicas que alg�n individuo de la empresa desempe�a con el fin de alcanzar un objetivo.\\
		
{\bfseries SEI}.- ``Software Engineering Institute'' de la Universidad Carnegie Mellon de Pittsburgh Pennsylvania, Estados Unidos. Es el responsable del desarrollo del CMM� - Capability Madurity Model.\\
		
{\bfseries Servidor}.- Ordenador remoto (en alg�n lugar de la red) cuya tarea es proporcionar informaci�n a los clientes seg�n peticiones.\\
		
{\bfseries SGML}.- Standard Generalized Markup Language. \\
		
{\bfseries Shell}.- Interface de un programa. \\
		
{\bfseries Smalltalk}.- Lenguaje de programaci�n Orientado a Objetos que naci� como fruto de una investigaci�n de XEROX en 1972.\\
		
{\bfseries Tester}.- Persona encargada de hacer las pruebas al software.\\
