\chapter{INTRODUCCI�N}
%\author{Telma C. Frege Issa}

El desarrollo de software ha evolucionado pasando del inicial estilo artesanal al industrial, donde los productos de software son desarrollados por equipos complejos de varios inform'aticos cuyo trabajo debe cumplir con fuertes exigencias presentadas por las normas de calidad. Las empresas de software abundan y tanto usuarios (consumidores) como desarrolladores (competencia) han hecho que el mercado se vuelva cada d'ia m'as exigente, obligando a todas las empresas del 'area a implementar metodolog'ias de control de calidad que les permitan ofrecer productos de software altamente competitivos.

\section{Planteamiento del problema}

La pol'itica de implementar metodolog'ias para el control de la calidad en las empresas de desarrollo del software es estrat'egica. En la actualidad la calidad de un producto es tan importante como su funcionalidad. De hecho, si la calidad no figura entre los requisitos de cualquier proyecto, muy probablemente 'este fracasar'a. La ausencia de m'etodos de control de calidad ocasiona en la mayor'ia de los casos retrasos en las entregas (desfases en los cronogramas), productos con fallas e incluso proyectos que nunca se concluyen, lo que deriva en problemas econ'omicos, derroche innecesario de recursos y, lo que es peor, en el deterioro de la imagen de la organizaci'on.\\

A su vez, el control de la calidad del software no es una tarea f'acil, requiere de expertos en el 'area y de instrumentos que les ayuden en su labor, por lo que surge la necesidad de contar con herramientas de apoyo en el proceso del Aseguramiento de la Calidad del Software (SQA).\\

Por otro lado, si bien ya existen herramientas (Test Director, QStat, QACenter Enterprise Edition, etc.) que ayudan al control de la calidad, 'estas poseen las siguientes falencias:
\begin{itemize}
\item Su uso es complicado (son poco amigables).
\item No son aptas para el trabajo en equipo, es decir, no permiten que m'as de un usuario las utilice al mismo tiempo.
\item Requieren que sus usuarios sean expertos en Calidad.
\item No se acomodan completamente a las necesidades de la empresa, como por ejemplo, el desarrollo distribuido y a distancia que es tan com'un actualmente.
\item Su costo es elevado.
\end{itemize}
	
\section{Objetivo general}

Desarrollar una herramienta orientada a la World Wide Web que apoye al control de la calidad en proyectos de desarrollo de software.

\section{Objetivos espec�ficos}

\begin{itemize}
\item Definir la manera en que el SQA puede ser sistematizado para poder ofrecer a los usuarios la posibilidad de implementar las t�cnicas de SQA de una manera sencilla.
\item Implementar la herramienta de manera que 'esta ofrezca a los usuarios la posibilidad de utilizar m'as de una t'ecnica de control de calidad, de una manera amigable.
\item Dotar a la herramienta con las funcionalidades necesarias para apoyar al usuario en el control de la calidad de un proyecto durante todas las etapas del desarrollo (desde los requerimientos hasta la prueba final).
\item Proporcionar un sistema y documentaci'on de base para los profesionales que quieran hacer un control serio de la calidad de sus productos durante su desarrollo a fin de asegurar que el mismo cumpla con las normas m'inimas y se tenga un buen empleo de los recursos.
\item Contribuir para que el control de la calidad del software deje de ser una teor�a ajena en nuestro medio y comience a ser implementado como una pr�ctica habitual en las empresas, de tal manera que �stas puedan ofrecer mejores productos.
\end{itemize}

\section{Alcances y limitaciones}

\subsection{Alcances}

\begin{itemize}
\item El trabajo contempla aspectos de investigaci'on y de implementaci'on.
\item La herramienta es amigable para el usuario.
\item La herramienta proporciona datos estad'isticos b'asicos de los resultados obtenidos al terminar el control de la calidad de un determinado proyecto.
\item El sistema est� orientado al Web; es una herramienta ASP (Application Server Provider), lo que permite el acceso a la misma mediante una simple conexi'on al servidor, adem'as de ser multiusuario y multiproyecto.
\item La herramienta cuenta con un soporte multi-idioma. Inicialmente est'a disponible en espa�ol e ingl'es.
\item	El sistema emite mensajes de alerta a los usuarios cuando �stos tienen tareas pendientes.
\end{itemize}
		
\subsection{Limitaciones}

\begin{itemize}
\item	El sistema no incluye el estudio ni la implementaci'on de las especificaciones de documentaci'on.
\item	{\sloppy El sistema es una herramienta de gesti'on, no cuenta con ning'un m'odulo inteligente que eval'ue de manera directa el cumplimiento de las m'etricas u otros criterios que son registrados en el mismo.}
\item	La herramienta no est� destinada a la administraci'on de proyectos, solamente ofrece un marco de trabajo b'asico a este nivel.
\item La herramienta fue implementada utilizando la arquitectura MVC (Model-View-Controller).
\end{itemize}

\section{Justificaci'on}

La falta de 'enfasis en el tema de Control de Calidad durante la formaci'on de los ingenieros en el pa'is ocasiona que se le reste importancia a este aspecto y, por lo tanto, la mayor'ia de los productos no alcanzan el nivel de competitividad exigido  por otras empresas del extranjero.\\

Tanto el documento como la herramienta ofrecer'an a las empresas de software la oportunidad de hacer un control de la calidad de sus productos durante su desarrollo y, por consiguiente, las beneficia con respecto a la competencia y a sus clientes.

\clearpage 
